\documentclass[a4paper, 14pt]{extarticle}

% кодировка
\usepackage[utf8]{inputenc}
\usepackage[T2A]{fontenc}

% поля
\usepackage[left=30mm,right=15mm,top=20mm,bottom=20mm]{geometry}
\usepackage{fancyhdr}

% переносы слов
\usepackage[english,russian]{babel}

% шрифт Таймс
\usepackage{tempora}
\usepackage{newtxmath}

% межстрочный интервал
\usepackage[onehalfspacing]{setspace}

% отступ первой строки
\usepackage{indentfirst}
\setlength{\parindent}{1.25cm}

\pagenumbering{arabic}

\usepackage{tabularx}
\usepackage{amsmath}

\makeatletter
\def\@discipline{---}
\newcommand{\discipline}[1]{
  \def\@discipline{#1}
}
\def\@group{---}
\newcommand{\group}[1]{
  \def\@group{#1}
}
\def\@email{---}
\newcommand{\email}[1]{
  \def\@email{#1}
}
\def\@code{---}
\newcommand{\code}[1]{
  \def\@code{#1}
}
\def\@version{1}
\newcommand{\version}[1]{
  \def\@version{#1}
}
\def\@draft{0}
\newcommand{\draft}[1]{
  \def\@draft{#1}
}
\renewcommand{\title}[1]{
  \def\@title{#1}
  \let\doctitle\@title
}
\renewcommand{\date}[1]{
  \def\@date{#1}
  \let\docdate\@date
}
\makeatother


\usepackage{auto-pst-pdf}
\usepackage{pstricks}
\usepackage{pst-barcode}
\usepackage{xcolor}
\usepackage{graphicx}

% подписи
\usepackage[singlelinecheck=false]{caption}
\DeclareCaptionLabelSeparator{gost}{~---~}
\captionsetup{labelsep=gost}

% иллюстрация
\newcommand{\fig}[3][1]{
    \begin{figure}[h]
        \centering
        \includegraphics[width=#1\textwidth]{#2}
        \caption{#3}\label{#2}
    \end{figure}
}
\renewcommand{\thefigure}{\arabic{figure}}
\DeclareCaptionLabelFormat{gostfigure}{Рисунок #2}
\captionsetup[figure]{justification=centering, labelformat=gostfigure, position=bottom}

\usepackage[ddmmyyyy,hhmmss]{datetime}
\usepackage{etoolbox}

\definecolor{verylightgray}{gray}{0.9}

\makeatletter

\fancypagestyle{plain}{
    \fancyhf{}
    \fancyhead[R]{\ifnum\value{page}>1 \@discipline \else \fi}
    \fancyfoot[L]{\texttt{\textcolor{verylightgray}{\@code~\ifdefempty{\@version}{}{v\@version}}}}
    \fancyfoot[C]{\thepage}
    \fancyhead[L]{\ifnum\@draft=1 \footnotesize\texttt{Драфт от \docdate{} Версия \@version{} Экспортировано \today{} в \currenttime{} UTC} \else {\ifnum\value{page}>1 \doctitle \else \textcolor{gray}{\small{Экспортировано \today{} в \currenttime{} UTC}} \fi} \fi}
}

\newcolumntype{R}{>{\raggedleft\let\newline\\\arraybackslash\hspace{0pt}}X}

\def\@maketitle{
    \centering
    \begin{tabularx}{\linewidth}{X R}
        {\LARGE \@title} & Дисциплина: \@discipline\\
        \@author & \@email\\
        \@date & \@group\\
        \vspace{1.5em}\\
        \hline\\
    \end{tabularx}
}

\makeatother

\pagestyle{plain}

% Augmented matrix style
% Usage: *matrix environment with [ccc|c]-like argument
\makeatletter
\renewcommand*\env@matrix[1][*\c@MaxMatrixCols c]{%
  \hskip -\arraycolsep
  \let\@ifnextchar\new@ifnextchar
  \array{#1}}
\makeatother

\usepackage{hyperref}
\hypersetup{
    colorlinks,
    linkcolor=black,
}

\title{Сборник лекций}
\discipline{Этика делового общения}
\code{BCE2101-DC}
\version{2.0.1}
\date{06/05/2023}
\draft{0}
\author{}
\email{}
\group{БФИ2202}

\def \missed{\textit{<пропущена часть предложения>}}

% For compatibility with imported 1,2 lectures
\usepackage{multienum}
\newtheorem{definition}{Определение}

\begin{document}

\maketitle

При подготовке сборника использованы записи лекций от\\@danil\_sidoruk~(1-2), @ohrimmm~(3), @Sugarr\_0k~(4) и @alozitskiy~(5-7).

Некоторые слова или фрагменты могут быть пропущены: \missed{}.
Некоторые фрагменты \textit{заполнены по смыслу (и выделены курсивом)}, но могут быть неточными.

Все предложения по дополнению или исправлению сборника можно направлять на почту \href{mailto:losica2014+bce@gmail.com?subject=%D0%94%D0%BE%D0%BF%D0%BE%D0%BB%D0%BD%D0%B5%D0%BD%D0%B8%D0%B5%20%D0%BA%20%D1%81%D0%B1%D0%BE%D1%80%D0%BD%D0%B8%D0%BA%D1%83%20%D0%BB%D0%B5%D0%BA%D1%86%D0%B8%D0%B9%20(%D0%AD%D0%94%D0%9E)&body=%D0%9E%D0%BF%D0%B8%D1%88%D0%B8%D1%82%D0%B5%20%D0%B7%D0%B4%D0%B5%D1%81%D1%8C%2C%20%D1%87%D1%82%D0%BE%20%D0%B2%D1%8B%20%D1%85%D0%BE%D1%82%D0%B8%D1%82%D0%B5%20%D0%B4%D0%BE%D0%BF%D0%BE%D0%BB%D0%BD%D0%B8%D1%82%D1%8C%20%D0%B8%D0%BB%D0%B8%20%D0%B8%D1%81%D0%BF%D1%80%D0%B0%D0%B2%D0%B8%D1%82%D1%8C.%20%D0%A3%D0%BA%D0%B0%D0%B6%D0%B8%D1%82%D0%B5%20%D1%82%D0%B5%D0%BC%D1%83%2C%20%D0%BC%D0%B5%D1%81%D1%82%D0%BE%20(%D0%BD%D0%BE%D0%BC%D0%B5%D1%80%20%D1%80%D0%B0%D0%B7%D0%B4%D0%B5%D0%BB%D0%B0)%20%D0%B8%20%D0%B2%D0%B0%D1%88%D0%B5%20%D0%B8%D1%81%D0%BF%D1%80%D0%B0%D0%B2%D0%BB%D0%B5%D0%BD%D0%B8%D0%B5.%20%D0%9F%D0%BE%20%D0%B2%D0%BE%D0%B7%D0%BC%D0%BE%D0%B6%D0%BD%D0%BE%D1%81%D1%82%D0%B8%20%D0%BF%D1%80%D0%BE%D0%BA%D0%BE%D0%BC%D0%BC%D0%B5%D0%BD%D1%82%D0%B8%D1%80%D1%83%D0%B9%D1%82%D0%B5%2C%20%D0%BF%D0%BE%D1%87%D0%B5%D0%BC%D1%83%20%D0%B2%D1%8B%20%D1%85%D0%BE%D1%82%D0%B8%D1%82%D0%B5%20%D1%82%D0%B0%D0%BA%20%D1%81%D0%B4%D0%B5%D0%BB%D0%B0%D1%82%D1%8C.}{losica2014+bce@gmail.com}.

Актуальная версия сборника доступна по адресу \url{https://losica2014.github.io/bce2101-dc/} или \url{https://github.com/losica2014/bce2101-dc}.

\section*{Список изменений}
\begin{description}
    \item[Версия 2.0.1] Исправлена грамматика.
    \item[Версия 2] Добавлены лекции 1-5, 7.
    \item[Версия 1] Добавлена лекция 6. 
\end{description}

\clearpage

\tableofcontents
\clearpage

\section{Понятие деловой этики, корпоративная культура как подсистема деловой культуры. Лекция от 02.02.2023 г.}

\subsection{Понятие деловой этики}

Термин «этика» относится к древнейшей науке, и относится к \textbf{5–4 векам до нашей эры}.

Родоначальником данного термина является \textbf{Аристотель}. Он образовал понятие «этический» (добродетель). Аристотеля занимала философская проблема: \textbf{«как следует поступать человеку, чтобы жить добродетельно?»}

\begin{definition}
\textbf{Этика} — область философии, предметом изучения которой являются проблемы морали

\textbf{Этика} — совокупность представлений о том, что есть добро, зло, справедливость, добродетель, нравственность

\textbf{Этика} — нравственные нормы, характеризующие ценности и нормы поведения определенных профессиональных групп
\end{definition}

Слово мораль относится к \textbf{философскому творчеству} и рассматривается непосредственно им. Проблемами этики и морали также занимался \textbf{Цицерон} Мораль происходит от слова \textbf{нравственный} и чаще всего понимается как \textbf{форма назидания в отношении норм поведения человека}. Этика рассматривается как \textbf{важнейшая система нормативной регуляции поведения человека в обществе}. Мораль рассматривается как синоним понятия \textbf{этика} и \textbf{нравственность}

\hfill

Важнейшим условием морального поведения служит присутствие в сознании индивида \textbf{нравственного идеала}

Этика \textbf{сформировалась} в эпоху, когда начал происходить распад родоплеменного строя и формирование раннего рабовладения.

К этому моменту общество накопило \textbf{обширные представления о добре и зле}. Эти представления \textbf{были выражены в древних эпосах}, например в \textbf{«Сказании о Гильгамеше»}

Также рассматривался вопрос смерти и бессмертия. Считалось, что \textbf{бессмертие человека — дела, которые он оставил после себя}

Учение вывели современное определение деловой этики (этики делового общения)

\begin{definition}
\textbf{Деловая этика} — это совокупность нравственных норм, правил и представлений, обеспечивающих регуляцию поведения людей в процессе производственной деятельности
\end{definition}

Деловая этика складывалась исторически в рамках традиционного общества, \textbf{определялась социальной структурой}, для которой была характерна роль \textbf{ритуалов, традиций, обычаев} и так далее.

Таким образом, \textbf{в традиционном обществе еще нет разрыва} между общественными моральными нормами и нормами деловой этики. \textbf{Этические ценности обладают самодостаточным значением}

Уже в Древней Греции заметное внимание уделялось изучению проблем деловой этики. Сократ также занимался вопросами этики, и считал, что \textbf{в основе делового общения лежит экономическая потребность}. Соответственно, для древних греков \textbf{статус начальника представлялся значительно выше, чем статус подчиненного}

\hfill

Для делового человека нового времени \textbf{характерно состояние внутренней психологической раздвоенности}: с одной стороны \textbf{он стремится к финансовой прибыли}, с другой стороны \textbf{он несет ответственность перед обществом (государством)}

Известный историк и экономист \textbf{Макс Вебер} также рассматривал вопросы этики, и считал, что \textbf{в основе раннего буржуазного общества лежит дух капитализма}

Для психологии современного делового человека \textbf{остается актуальна не только внутренняя раздвоенность}, но также и \textbf{проблема самоидентификации}

Известный философ Эрих Фромм в своей книге «Бегство от свободы» \textbf{определяет наличие этики в современном обществе}, а также \textbf{определяет человеческий тип как «личность с рыночным характером»}

В \textbf{современном крупном бизнесе} распространены \textbf{два кардинально разных подхода к вопросам деловой этики}:

\begin{enumerate}
    \item Сторонники \textbf{морального прагматизма} полагают, что \textbf{в бизнесе следование нормам морали лишь вредит интересам дела}. То есть, в данной среде тщательно избегают разговоров на темы нравственности и морали
    \item Сторонники противоположного подхода считают, что \textbf{нормы морали обязательны к исполнению в деловой сфере}. Очень часто данная позиция благотворно влияет на имидж и бренд компании, что в конечном счете приводит к ее финансовому процветанию
\end{enumerate}

\subsubsection{Основные категории этики}

\begin{multienumerate}
    \mitemxxx{Добро}{Зло}{Долг}
    \mitemxxx{Ответственность}{Честь}{Совесть}
    \mitemxxx{Достоинство}{Равенство}{Справедливость}
\end{multienumerate}

\textbf{Категориями} называются понятия, в которых \textbf{этика раскрывает сущность морали и моральной деятельности человека}

\textbf{Нормы} формируются на уровне общественного сознания, и это отражается категориями \textbf{добро} и \textbf{зло}

Такие категории как \textbf{совесть}, \textbf{честь}, \textbf{достоинство} характеризуют уровень \textbf{индивидуального морального сознания}

\subsection{Корпоративная культура как плод системы деловой культуры}

С конца 90-х годов XX века понятие \textbf{корпоративная культура} прочно вошло в лексикон отечественного бизнеса.

\subsubsection{Элементы корпоративной культуры}

\begin{multienumerate}
    \mitemxx{Приветствуемый стиль общения (в данной организации)}{Системность и регулярность менеджмента}
    \mitemxx{Система вознаграждений и поощрений}{Декларируемые ценности}
    \mitemxx{Регламентирующие документы}{Наличие и способы реализации корпоративных мероприятий}
\end{multienumerate}

\subsubsection{Уровни корпоративной культуры}

Понятие \textbf{уровней корпоративной культуры} впервые ввел известный ученый Эдгар Шейн в 1981 году.

Он выделяет \textbf{три уровня корпоративной культуры}:

\begin{enumerate}
    \item Поверхностный — внешние факты
    \item Внутренний — ценностные ориентации и верования
    \item Глубинный — базовые предположения
\end{enumerate}

\subsubsection{Факторы, влияющие на корпоративную культуру}

\textbf{Факторы, влияющие на корпоративную культуру}:

\begin{multienumerate}
    \mitemxx{Личности людей}{Личности топ-менеджмента и ключевых сотрудников}
    \mitemx{Внешнее окружение}
\end{multienumerate}

\subsubsection{Принципы формирования корпоративной культуры}

Основные \textbf{принципы формирования корпоративной культуры}:

\begin{multienumerate}
    \mitemxxx{Свобода}{Справедливость}{Общечеловеческие духовные ценности}
\end{multienumerate}

\textbf{Неэффективные} меры формирования корпоративной культуры:

\begin{enumerate}
    \item Административное насаждение правил и норм: введение системы штрафов, чрезмерный контроль за сотрудниками, устрашающие меры и так далее
    \item Назначение ответственных за создание корпоративной культуры
    \item Привлечение внешних специалистов
\end{enumerate}

Некоторые приемы реализации корпоративной культуры разрабатываются в различных организациях. Например, \textbf{размещение ценностей корпоративной культуры}, \textbf{особые традиции в компании}, \textbf{методы вдохновления сотрудников}, \textbf{обучение персонала профессиональным навыкам}

\subsubsection{Принципы ведения дел в России}

В корпоративной культуре выделяют \textbf{двенадцать принципов ведения дел в России}:

\begin{enumerate}
    \item \textbf{Принципы личности}:
    \begin{enumerate}
        \item Прибыль важнее всего, но честь дороже прибыли
        \item Уважай участников общего дела
        \item Воздерживайся от насилия или угрозы его применения
    \end{enumerate}
    \item \textbf{Принципы профессионала}:
    \begin{enumerate}
        \item Всегда веди дело сообразно средствам
        \item Опрадывай доверие — в нем ключ к успеху
        \item Конкурируй достойно
    \end{enumerate}
    \item \textbf{Принципы гражданина России}:
    \begin{enumerate}
        \item Соблюдай действующие законы и подчиняйся законной власти
        \item Для законного влияния объединяйся с единомышленниками
        \item Твори добро для людей и ничего не требуй за это взамен
    \end{enumerate}
    \item \textbf{Принципы гражданина земли}:
    \begin{enumerate}
        \item При создании и ведении дел не причиняй ущерба природе
        \item Найди в себе силы противостоять преступности и коррупции
        \item Проявляй терпимость к представителям другим культур, верований и стран
    \end{enumerate}
\end{enumerate}

\section{Деловое общение. Коммуникации в деловом общении. Лекция от 16.02.2023 г.}

\subsection{Деловое общение}

\begin{definition}
    \textbf{Деловое общение} — процесс взаимосвязи и взаимодействия субъектов, при котором осуществляется обмен деятельностью, информации, опыта; целями которого являются:
    \begin{multienumerate}
        \mitemxx{Решение конкретной задачи}{Разрешение определенной проблемы}
        \mitemx{Достижения какой-то заявленной определенной цели}
    \end{multienumerate}
\end{definition}

Партнер в деловом общении выступает как личность, для всех его участников, которых отличает компетентность и взаимопонимание в обсуждаемых вопросах

Таким образом, главной целью делового общения является взаимовыгодное сотрудничество сторон

Основные формы делового общения:
\begin{multienumerate}
    \mitemxxx{Деловая беседа (возможно по телефону)}{Деловые переговоры}{Служебное совещание}
    \mitemxxx{Деловая дискуссия}{Пресс-конференция}{Публичная речь}
    \mitemx{Деловая переписка}
\end{multienumerate}

Ученые выдвигают технологию делового общения (то есть, эффективную общепринятую модель взаимодействия сторон при реализации форм современного делового общения)

Выделяют следующие технологии делового общения:
\begin{multienumerate}
    \mitemx{Вербальное деловое общение — при данной форме общения используется устная или письменная передача информации}
    \mitemx{Невербальное деловое общение — при передаче информации используется язык жестов, мимики, позы тела и так далее}
    \mitemx{Дистанционное деловое общение — общение посредством почты, телефона и так далее}
\end{multienumerate}

В результате делового общения происходит взаимодействие групп людей, и в науке выделяют следующие функции делового общения:

\begin{multienumerate}
    \mitemxxx{Информационно-коммуникативная — участники переговоров обмениваются определенной информацией}{Интерактивная — связана с процессом взаимодействия между участниками делового общения}{Перцептивная — выражается через процесс восприятия одного человека другим}
\end{multienumerate}

\subsubsection{Стили общения}

\begin{definition}
    Под \textbf{стилем общения} понимают комплекс личных качеств и действий лица, который в значительной мере предопределяет его отношение к определенной жизненной ситуации
\end{definition}

С этой точки зрения выделяют \textbf{ритуальный}, \textbf{манипулятивный} и \textbf{гуманистический} стили общения

\hfill

\textbf{Ритуальное общение} связано со стремлением индивида все сделать по определенным правилам и нормам

При \textbf{манипулятивном общении} партнер рассматривает своего собеседника в качестве инструмента для достижения определенной цели

\textbf{Гуманистическое общение} предполагает совместный поиск ответов на сложные вопросы. На первый план при этом стиле общения выходят духовные ценности

\hfill

Ключевую роль в деловом мире играет \textbf{невербальное} общение. Оно связано с особенностями человеческой психики и фиксирует наше внимание на внешних проявлениях собеседника

Именно невербальное общение как бы скрывает наиболее важную информацию в процессе деловых переговоров

Также существует \textbf{визуальное общение} — сфера контакта глаз. Визуальный контакт играет главную роль в невербальном общении

\subsubsection{Основные принципы делового общения}


\begin{multienumerate}
    \mitemxx{\textbf{Деловая репутация}. Роль деловой репутации огромна, поскольку она нарабатывается годами и активно влияет на деловой успех предпринимателя}{\textbf{Конкретность и четкость}}
    \mitemxx{\textbf{Взаимовыгодное сотрудничество}}{\textbf{Контроль над ситуацией}. Важно уметь держать себя в руках. Очень важно уметь контролировать свое поведение и бизнес}
    \mitemxx{\textbf{Умение слышать}. Необходимо инвестировать средства не только в сферу производства, но и в сферу обслуживания. Особенно важно \textbf{умение слушать и слышать своего клиента}}{\textbf{Умение сосредоточиться на главном}. Предпринимателю очень важно быстро и правильно расставлять приоритеты в своей деятельности}
    \mitemx{\textbf{Умение отделить личные отношения от бизнеса} считается одним из важных принципов деловой жизни}
    \mitemx{\textbf{Умение быть честным}. Этот принцип состоит в том, что необходимо соблюдать заключенные договоры и нести за них ответственность. Ради заключения какой-либо выгодной сделки нельзя манипулировать другими людьми, скрывать правду и так далее}
\end{multienumerate}

\subsection{Деловое общение как коммуникация}

\textbf{Деловое общение как коммуникации} — взаимодействие двух или более людей, целью которых является решение какой-либо проблемы

Деловая коммуникация представляет собой процесс \textbf{целесообразный} — то есть, вступая в контакт, коммуниканты преследуют определенные цели и интересы, которые могут совпадать между собой или, наоборот, вступать в противоречие

\hfill

В науке существует такое понятие как \textbf{коммуникативная компетентность} — она включает в себя умение адекватно вести себя в определенной ситуации и ставить определенные цели

Наиболее благоприятной формой коммуникации для убеждения собеседника является деловая беседа

\subsubsection{Типология барьеров в коммуникации}

\begin{enumerate}
    \item \textbf{Барьеры, обусловленные факторами среды}, то есть характеристики внешней физической среды, например, шум в помещении, за окном, ремонтные работы и так далее

    Негативное влияние усиливается, если в помещении плохая акустика

    Отвлекающей также может быть окружающая обстановка: яркое солнце, тусклый свет, цвет стен в помещении, пейзаж за окном и так далее

    А также температурные условия
    \item \textbf{Технические барьеры}: плохая телефонная связь, помехи в радиоэфире и так далее

    Технические барьеры в коммуникации связаны как с работой технических средств, так и могут быть обусловлены человеческим фактором 
    \item \textbf{Человеческие барьеры} — барьеры, причиной которых является сам человек. Например, эмоциональные барьеры, фонетический барьер (когда участники общения говорят на различных языках и диалектах)
    \item \textbf{Семантические барьеры} — проблема использования различных жаргонов, сленгов
    \item \textbf{Стилистический барьер} возникает при несоответствии стиля речи коммуникатора; тогда, когда информация передается функционально-книжным языком
\end{enumerate}

\subsubsection{Структурирование информации в деловом взаимодействии}

Существует \textbf{два основных приема структурирования информации в деловом взаимодействии}:

\begin{enumerate}
    \item Суть \textbf{правила рамки} состоит в том, что начало и конец любого делового разговора должны быть четко очерчены. То есть, в начале как правило сообщаются цели, намерения, перспективы, возможные результаты; а в конце обязательно должны быть подведены итоги, сделаны выводы

    Но в повседневном общении правило рамки достаточно часто нарушается
    \item \textbf{Правило цепи} направлено на внешнее структурирование общения. То есть, необходимые сведения должны так быть выстроены, что как бы соединены в цепь по каким-либо признакам
\end{enumerate}
    
Ученые доказали, что лучше всего запоминается фраза, состоящая из четырех—четырнадцати слов

\subsubsection{Вывод}

Для всех людей важно общаться таким образом, чтобы их правильно понимали; чтобы их слова не наталкивались на стену непонимания; чтобы их слушали и слышали

\section{Деловое общение как взаимодействие. Лекция от 02.03.2023 г.}

Специфика делового общения заключается в том, что столкновением взаимодействия экономических интересов и социальное регулирование осуществляется в правовых рамках. Чаще всего люди вступают в деловые отношения, чтобы юридически оформить взаимодействие в той или иной сфере.

В зависимости от различных признаков деловое общение делится на:
\begin{itemize}
    \item Устное/письменное общение
    \item Диалогическое/Монологическое
    \item Межличностная/Публичная
    \item Непосредственное/Опосредованное
    \item Контактное/Дистантное
\end{itemize}

Деловое поведение – это осуществляемое в условиях рынка взаимодействие деловых людей с внешней средой (в основном с деловыми партнерами).

Характер и особенности данного взаимодействия определяются спецификой рыночной среды.

\subsection{Теории межличностного взаимодействия.}

Существует несколько теорий, объясняющих межличностное взаимодействие:
\begin{itemize}
    \item Теория обмена
    \item Символический интеракционизм
    \item Теория управления впечатлениями
    \item Психоаналитическая теория
\end{itemize}

\subsubsection{Теория обмена}
Согласно теории обмена каждый из нас стремится уравновесить вознаграждение и  затраты, чтобы сделать наше взаимодействие устойчивым и приятным.
Эту теорию разработал известный американский социолог Джордж Хоманс.
Согласно его теории люди стремятся взаимодействовать друг с другом только в том случае, когда в процессе взаимодействия происходит взаимовыгодный обмен важными ресурсами.

Хоманс выделяет основные принципы практической реализации теории обмена:
\begin{itemize}
    \item Поведение человека способствует или препятствует эффективному межличностному взаимодействию 
    \item В ситуации, при которой вознаграждение за поведение зависит от определенных условий, человек будет стремиться воссоздать, чтобы эта ситуация была благоприятна для него.
    \item Уровень поощрения за реализацию взаимодействия
\end{itemize}

\subsubsection{Теория символического интеракционизма Джорджа Мида.}
Мид рассматривал любые действия человека, как социальное поведение, которое всегда основано на коммуникации. По его мнению люди все время взаимодействуют друг с другом, даже в отсутствии прямого контакта.
По его заключениям люди реагируют не только на конкретные действия людей, но и на их мысленные намерения.
Мид подчеркивал, что сущность символического интеракционизма заключается в том, что взаимодействие между людьми рассматриваются как непрерывный диалог, в процессе которого они наблюдают, осмысливают намерения друг друга и реагируют на них.

Таким образом, центральная идея данной концепции: Личность формируется всегда во взаимодействии с другими личностями.

\subsubsection{Теория управления впечатлениями.}
Разработал Гоффман. Данный социолог жил в 20 веке.

По его мнению люди самостоятельно создают определенные ситуации, которые оказывают благоприятное впечатление на других людей с целью осуществить что-то желаемое.
По мнению Гоффмана социальные ситуации следует рассматривать как драматические спектакли в миниатюре.
Т.е. люди ведут себя подобно актерам на сцене, используют определенные декорации чтобы создать впечатления о себе.


\subsubsection{Психоанализ Зигмунда Фрейда.}
В его идеи лежит процесс взаимодействия опыта участников, связанных детскими переживаниями.
Фрейд говорил, что люди используют понятия, которые они освоили в раннем детстве.
Для людей очень характерно почтительное отношение к человеку, который наделен финансовой, политической, экономической, организационной и другой властью.
Также Фрейд выводит теорию покорности по отношению к лидеру.

\subsubsection{Этнометодология Гарфинкеля.}
Суть теории:  он рассматривает обыденные нормы, правила поведения смысла его в контексте повседневного социального взаимодействия людей.
Этнометодология обязательно использует методы социологического исследования, которая связана с описанием повседневных практик, социальных взаимодействий.
Горфинкель первостепенное значение придавал социальной среде и взаимодействию людей.
А также Горфинкель исследовал разные ситуации поведения людей. Например: поведение людей в суде, обычные беседы. 


\subsubsection{Транзакционный анализ Эрика Берна.}
Известная его книга «Люди и игры», широко используется в психотерапии, когда у людей наблюдаются психические расстройства.
Психологи также активно используют ее, например, по коррекции поведения человека.

Эрик Берн вводит понятие «транзактный анализ – анализ взаимодействия».
Ученый заметил, что мы в различных ситуациях занимаем различные позиции по отношению друг к другу. У каждого своя определенная позиция. Таким образом, суть теории Эрика Берна сводится к тому, что когда ролевые позиции партнеров согласованы, то их взаимодействие будет доставлять им чувство удовлетворения. Если есть несогласованность, то возникает конфликт.

\subsection{Манеры общения.}
Каждый человек обладает своей своеобразной, неповторимой манерой общения.
Типы:
\begin{itemize}
    \item Доминантный собеседник – человек жесткий, напористый. Легко перебивает других людей.
    \item Недоминантный собеседник – человек уступчивый, легко теряется. Никогда не позволит себе перебить другого собеседника. 
    \item Мобильный собеседник – всегда с легкостью переключается с других занятий на общение. Говорит быстро, иной раз торопливо. Проститься с ним также легко, как и завести беседу
    \item Ригидный  - такому собеседнику некоторое время, чтобы включить в беседу. Он всегда внимательно слушает, основателен, говорит неспешно. Мысли излагает очень подробно. Попрощаться с ним сразу невозможно 
    \item Экстраверт – этот человек всегда расположен к общению. Без общения скучает. Всегда любопытен ко всем сторонам жизни людей, даже своя жизнь не так интересует, как жизнь окружающих.
    \item Интроверт  - не склонен к внешней коммуникации, для общения обычно выбирает до трех собеседников, которые обычно похожи на него самого.
\end{itemize}

\subsection{Характеристика стратегий межличностного взаимодействия.}
В повседневной жизни человек вступает во взаимодействие с другими людьми.
Обязательно должен быть мотив. Ученые выделяют следующие мотивы:
\begin{itemize}
    \item Мотив общего выигрыша (мотив кооперации) 
    \item Мотив собственного выигрыша 
    \item Мотив относительного выигрыша 
    \item Мотив выигрыша другого
    \item Мотив различий в выигрышах
\end{itemize}

\section{Деловой этикет. Лекция от 16.03.2023 г.}

Слово этикет французского происхождения, означает соблюдение норм и правил.

В повседневной жизни человек, владеющий этикетом в любой ситуации способен найти оптимальную линию поведения. Знание делового этикета, необходимое профессиональное качество любого бизнесмена. 
Особенно важен этикет при общении с зарубежными коллегами. А также знание этикета основа делового успеха.

В Европе деловой этикет зародился в эпоху абсолютизма, а в России понятие этикет вошло в начале 18 века (правление Петра 1). Но правила поведения, которые необходимо соблюдать зародились еще в Киевской Руси. В частности, они излагались в известном произведении Владимира Мономаха «Поучение детям», где он описывает нормы этикета, как нужно вести себя за столом, что старшие должны заботиться о младших, младшие уважать старших и т.д.

Этикет – это система порядков, правил и норм социально-ролевого общения.

Функции этикета:
\begin{itemize}
\item Регламентирующая 
\item Символическая
\item Коммуникативная
\end{itemize}

А также специалисты выделяют несколько видов этикета:
\begin{itemize}
\item Светский (равенство субъектов, которые вступают в общение)
\item Деловой (правила, принятые в сфере делового общения)
\item Служебный (служебная иерархия (подчиненный и начальник))
\item Дипломатический
\item Профессиональный и т.д. 
\end{itemize}

Выделяют составляющие этикета:
\begin{itemize}
\item Внешний облик и одежда
\item Манеры 
\item Правила поведения
\item Культура речи
\end{itemize}

\subsection{Международный этикет.}

Существует международный этикет, понятие возникло достаточно недавно.

Культурные нормы живут благодаря передаче традиций.

Обязательное и основное правило этикета – искреннее уважение к другому человеку (сказали это оооочень важно). Это значит, что при первой встрече проявлять уважение к другому человеку, особенно если незнакомый, нельзя проявлять подозрительность и недоверие, другой человек будет это тонко чувствовать.

\subsubsection{Речевой этикет.}

Существует вербальный этикет (словарный запас, манеры, стилистика речи). Важная норма речевого этикета – это готовность всегда ответить на заданный вопрос. 

К речевым этикетным нормам относится:
\begin{enumerate}
\item Обращение (должно быть всегда правильно выбранной формы, тональности, энергетики голоса, от этого во многом зависит дальнейшее взаимопонимание между людьми, к человеку следует обращаться по имени, при первой встрече обязательно запоминать имя собеседника, так вы быстрее его к себе расположите, но в России сохраняется обращение по имени и отчеству, у американцев практикуется обращение по имени, но нужно предупредить собеседника об этом, у немцев обращение либо по фамилии, либо по титулу. Если вы уже знакомы с деловым партнером (он старый друг), можно с ним разговаривать как с другом, но не переходить границу и не обижать его. В повседневной жизни, обращения могут быть самыми разнообразными, главное, чтобы они были не оскорбительными для человека. В настоящее время наиболее распространенными формами обращения к аудитории являются: дамы и господа, господа, уважаемые коллеги, дорогие друзья и т.д. Также можно, обращаясь к официальному лицу, немного повышать его в должности (заместителя министра назвать господин министр) ему это будет приятно. Следует быть особенно внимательным в странах, где при обращении сохраняются дворянские титулы (особенно это касается Англии)).
\item Приветствие (первым здоровается мужчина с женщиной, дальше здоровается младший со старшим, подчиненный с начальником и т.д. На официальных приемах в первую очередь всегда приветствуют хозяйку и хозяина, затем дам, после этого старших по положению мужчин и затем остальных. Сидящий мужчина, когда приветствует даму или человека старше по званию, должен обязательно встать)
\item Жесты, сопровождающие приветствие (наиболее распространенное рукопожатие (первой руку протягивает женщина мужчине, старший младшему, начальник подчиненному), поднятие руки, кивок головы, наклон, у женщин целуют ручку. Если мужчина приветствует на улице знакомую девушку, он должен немного приподнять свой головной убор, можно не приподнимать при -25. Если приветствие сопровождается рукопожатием, мужчина обязательно должен снять перчатку, женщина может не снимать, так как перчатка для женщины – часть женского туалета. Большое значение при приветствии имеет манера держаться (например, нельзя протягивая руку для приветствия, вторую держать в кармане, отводить глаза в сторону или продолжать разговор с другим человеком, это считается невоспитанностью и невежливостью). Не нужно шумно приветствовать собеседника, это некрасиво. Желательно, чтобы приветствие было развернутым и открытым, например: Добрый день Татьяна, как дела? )
\item Представление (младший представляет старший, мужчину женщина. Существуют два способа знакомства: Знакомство через посредника – вас представляют. Второй способ Самостоятельное знакомство. Если это официальное знакомство, обязательно нужно указать профессию человека, положение и должность. Если знакомство без посредников нужно обращаться так: Разрешите (позвольте) с вами познакомиться (вам представиться). Если молодежная среда, то обычно достаточно сказать только имя. Если официальная встреча обязательно либо фамилию, либо фамилию и имя.)
\item Комплимент (Приятные слова, несколько преувеличивающие положительные качества собеседника. Комплимент разделяют на два вида: светский (комплимент внешности и достоинств человека, обычно говорят знакомые, родственники и близкие люди. Считается, что сделать комплимент женщине проще чем мужчине. Комплимент всегда подчеркивает достоинство собеседника. Распространен в неофициальной обстановке. Всегда нужно благодарить за комплимент.) и деловой (обмен любезностями между сторонами партнерами. Взаимная обязательная процедура в деловой беседе. Существует письменный деловой этикет (тут понятно).) Комплимент нужно говорить обязательно, но он должен быть правдивым, иначе вас заподозрят в неискренности.)
\item Поздравление (признание значимости события или партнера. Поводы могут быть разными. Поздравление – знак внимания к человеку или организации.)
\item Сочувствие (Может быть по поводу болезни, утраты, смерти близких людей. Некоторые люди считают, что сочувствовать – это лишний раз напоминать об утрате. Но на самом деле всегда говорят, что людям приятно чувствовать поддержку в трудной ситуации. Ученные заметили закономерность – если мужчина находится в стрессовом состоянии, то он пытается побыть в одиночестве, некоторое количество времени, чтобы успокоиться. А женщина, наоборот, нуждается выговориться кому-нибудь и быстро в себя приходит)
\item Прощание (считается, что никогда нельзя прощаться навсегда (слово прощай лучше не говорить))
\item Подарки (Выражение отношения к событию. Бывают подарки руководителю, их нужно делать от коллектива и только по торжественному поводу. Если хотите лично сделать подарок руководителю, лучше делать это не перед коллективом, к вам может испортиться отношение, подумают, что вы пытаетесь выслужиться. Подарки коллегам по работе – принцип ты мне, я тебе. Также лучше всего в столе иметь запас открыток и безделушек, которые можно подарить коллеге по работе. Подарки клиентам – подарок от фирмы, обычно это продукция, которую выпускает эта фирма. В качестве подарка официальным лицам или деловым партнерам, можно подарить хорошую книгу, альбом с репродукциями картин известных художников и т.д. Иногда можно дарить дорогие спиртные напитки, главное знать вкус человека. Женщинам дарят цветы, принято дарить нечетное кол-во цветов. Возможны индивидуальные подарки, в зависимости от пристрастий человека. Обязательно поблагодарить человека за подарок. Если подарок очень дорогой, и вы не можете подарить потом что-то по ценности похожее, лучше отказаться от такого подарка (Благодарю вас, я не могу себе это позволить, не нужно объяснять почему). Если вы отправляете подарки через курьера, обязательно должна быть визитная карточка.)
\end{enumerate}

\subsubsection{Официальные приёмы.}

На переговоры нужно приходить официально одетыми.
Когда вы приезжаете на мероприятие, вы должны знать заранее, где вам сесть, как это сделать. Уважаемый гость всегда садиться напротив главного принимающего лица. Справа садиться первый заместитель, слева второй заместитель. Если переговоры проходят с участием переводчика, то рассадка такая: уважаемый гость напротив главного принимающего лица, справа первый заместитель, слева переводчик. Гости должны всегда сидеть лицом к входной двери, либо к окну.

Официальные приемы:
\begin{itemize}
\item дневные (менее торжественны, чем вечерние) (завтраки, между 12 и 13 ч.)
\item вечерние (``коктейль'' - между 17-18 \textit{(19-20 - прим. А.Л.)} часами вечера, длятся около 2 часов; ``ужин'' - после 21 ч.)
\end{itemize}

Обычно на приемах принято говорить тосты. Первый тост произносит хозяин приема. Дальше выступает главный гость. Остальные могут тосты не произносить.

Правила этикета за столом:
\begin{itemize}
\item Нельзя приступать к еде, пока этого не сделает хозяйка
\item Мужчины должны подождать пока к еде не притронутся дамы
\item После еды первой из-за стола встает хозяйка, потом все остальные
\end{itemize}

Желательно не опаздывать на мероприятия. 

\subsection{Деловая переписка.}

Существует деловая переписка – разновидность официальной переписки.

Обязательно обратить внимание на правильность написания фамилии и адреса человека, которому отправляете письмо. Деловые письма желательно писать на бланке организации. Подписывает такое письмо только руководство фирмы. Обязательно должна быть дата, день месяц год. Деловое письмо должно иметь безупречный внешний вид, быть на бумаге высшего качества, конверты для писем должны быть соответствующего размера и качества.

В письме должен прослеживаться доброжелательный тон. Негативную информацию лучше всего располагать в середине письма. Письмо должно быть написано либо на английском, либо на языке той страны куда оно отправляется. Письмо всегда складывается текстом внутрь.
Если отправляете поздравления или соболезнования, письмо обязательно пишется от руки.

\section{Особенности зарубежного делового этикета. Лекция от 30.03.2023 г.}

\subsection{Европейский деловой этикет}

\subsubsection{Англия}

Англичане характеризуются деловитостью, почитанием собственности, традиций, законопослушания. Считаются \missed{} в европейской \missed{} бизнеса.

В беседах ценят умение слушать, а в деловых отношениях пунктуальность. Уважение к собеседнику. Всегда соблюдают формальность. Общение по имени только после специального разрешения. Используются рукопожатия. Но английские бизнесмены мало уделяют времени подготовке переговоров. Он наблюдателен, психологичен, способен скрыть профессиональную неподготовку.

Одежда: женщины - костюмы/платья, мужчины - костюм с галстуком.

О делах после работы разговаривать не принято.

Приглашение на обед: если пригласили, то отказываться нельзя, приходить в смокинге/вечернем платье. Напитки: джин/виски. Не принято произносить тосты и чокаться.
Избегать разговоров о королевской семье, политике, финансах, личной жизни, \missed{}.
Чай только с молоком, а не сливками.
Также приглашают на ланч.

Если пригласили в дом - знак особого расположения.

Не принято обмениваться визитными карточками. Число 13 - часто отсутствует. Не принято дарить дорогие подарки.

\subsubsection{Франция}

Придают большое значение внешнему облику собеседника. На все официальные мероприятия мужчины - вечерние костюмы, женщины - вечерние платья; только из натуральных материалов и известных брендов. У женщин украшения только на мероприятия.

Ценят родную культуру, искусство.
Любят спорить. Не прощают пренебрежительного отношения.

Телефонные переговоры не приемлют. Переговоры начинаются с 11:00. Не пунктуальны. Приветствуют рукопожатия. Пользуются визитными карточками.
Не обращаться по имени (только если нет разрешения). Переговоры только на французском языке.
Не стремиться перейти на английский.

Французская кухня - предмет гордости. Не оставлять еду на тарелке. Подсаливать и добавлять специи (на столе) не принято.

Французы любят и предпочитают работать в одиночку. В разговоре не принято касаться тем религии, семейного положения, \missed{}

Если пригласили в дом, подарок: вино/шампанское.

Любят носить яркие рубашки. К деловому партнёру обращаются только на Вы. Деловые подарки не приняты.

\subsubsection{Германия}

Немцы бережливые, точные, свойственно гражданское мужество, сдержанность. Ценят профессионализм. Спешка неодобрительна.

Все встречи назначаются заранее. Проводятся с 10 до 16. Обращаться на Вы. Всегда присутствуют юристы. На всех руководящих постах пожилые люди.

Не принято приглашать делового партнёра домой, но если пригласили - большое уважение; подарок - вино, букет цветов (не розы) хозяйке.

Важны титулы, хозяйке титул мужа.

Пунктуальны. Не опаздывать. Не принято дарить деловые подарки.

Одежда строгая, мужчины - тёмный костюм/галстук; женщины - костюм (юбка, никаких брюк).

\subsubsection{Испания}

Человечные, с чувством юмора, способны работать в команде, эмоциональны.

В переговорах предпочитают на ты. Регламент встреч часто не соблюдается (любят говорить).

Приняты визитные карточки на испанском+английском.

Не принято приглашать домой, но если: подарок - цветы, конфеты и шампанское.

Не принято оставлять еду. Со спиртным - умеренность. Любят, когда хвалят их еду.

Хорошие собеседники, говорят на любую тему, но не принято: религия, режим Франко (2 мировая).

\subsection{Северо-американская деловая культура.}

\subsubsection{США}

Активные, \missed{}

В дружбе не постоянны, дружат по интересам.

Не любят слишком официальную атмосферу. Одежда - может быть повседневная. Переговоры от 30 минут до 1 часа. Решение могут и менять. Педантичны к оформлению документов.

Важна пунктуальность. Ценят трудолюбие, бережливость.

Считают, что знают всё деловое \missed{} других стран, что должны руководствоваться их (США) советами.

Способны бороться за доход, характерна напористость и агрессивность при заключении договоров.

\subsubsection{Канада}

Атмосфера бизнеса - европейская.

Пунктуальны. не любят, когда говорят о противоречиях между англо- и франкоязычным населением. Общительны о спорте.

Переговоры в ресторанах. Основные напитки - вино и пиво. При посещении дома - букет. Принято благодарить не за угощение, а за гостеприимство. Темы общения - любые, но не любят сравнение США и Канады. Обращаться на ты.

Одежда: высокого качества костюм/элегантная одежда.

\subsection{Деловая культура востока и арабских стран.}

Значительно более церемониальна.

\subsubsection{Китай}

Уделяют много внимания сбору информации о предмете переговоров. Часто затягивают переговоры.

Только личные встречи (никаких телефонов).

Назначаются переговоры за 3 месяца. Пунктуальны. Нельзя откладывать переговоры.

Допустимы: рукопожатие, поклон.

Характерна сдержанность, уважение к старшим.

При обращении помним, что в Китае фамилия впереди имени (обращение по фамилии).

Принят обмен визитками. Разрешены тосты. Практикуется вручение подарков. нельзя дарить часы.

\subsubsection{Япония}

Распространён коллективизм.

Носят чёрные (отличительный знак стажёров) или тёмно-синие костюмы.
Прикреплён значок с эмблемой места работы. Чем выше ранг, тем строже одежда.
Не следят за модой, главное в одежде - чистота (глаженность - не особо важна).

Принят обмен визитками (с небольшим поклоном).

Пунктуальны. Терпеливы. Стрессоустойчивы.

Приветствуются поклоны (меньше - рукопожатие). Чем ниже - тем больше уважения.

Рабочее время - больше всех в мире. Принято отказываться от отпуска.

Не принято приглашать домой, если вдруг - большое уважение. Не вытягивать руки/ноги. Не принято дарить цветы.

\section{Деловые мероприятия, деловые встречи и переговоры. Лекция от 13.04.2023 г.}

Деловые мероприятия - общественные события в бизнесе:
\begin{itemize}
    \item Конференции
    \item Презентации
    \item Круглые столы
    \item Форумы
    \item Семинары
    \item Встречи
\end{itemize}

\subsection{Деловая встреча}

Деловая встреча - форма взаимодействия между людьми, которая предполагает наличие конкретной значимой цели. Это организованный тип общения между людьми \missed{}.

\subsection{Пресс-мероприятия}

Пресс-мероприятия - встречи журналистов с представителями референтных групп (государственные учреждения, общественно-политические организации, коммерческие организации). Это эффективный метод передачи информации прессе и другим СМИ.

Причины созыва пресс-конференции:
\begin{itemize}
    \item Есть серьёзный информационный повод
    \item Необходимо, чтобы о новом событии узнала целевая аудитория
    \item Важно, что позиция компании по определённому \missed{}
    \item Если нужно, чтобы компания была упомянута в СМИ наряду с VIP-персоной
\end{itemize}

\subsection{Деловые переговоры}

Деловые переговоры обязательно проводятся по определённым правилам и подчиняются собственным закономерностям.
Главная цель - прийти к взаимовыгодному решению, избегнув конфликта.

Чтобы правильно сформулировать цели деловых переговоров надо знать:
\begin{itemize}
    \item Интересы организации
    \item Положение организации на рынке
    \item и т.д.
\end{itemize}

Деловые переговоры бывают:
\begin{itemize}
    \item Официальные (всё по протоколу) и неофициальные (непринуждённая беседа, не подписываются бумаги)
    \item Внешние (с деловыми партнёрами и клиентами) и внутренние (между сотрудниками)
\end{itemize}

Стадии переговоров:
\begin{enumerate}
    \item Подготовка
    \item Процесс переговоров
    \item Достижение согласия
\end{enumerate}

Считается, что если по итогу переговоров не был подписан контракт, то переговоров будто и не было.

Техники ведения переговоров:
\begin{enumerate}
    \item Тщательно избегать высказываний, оскорбляющих партнёра
    \item Не стоит игнорировать позицию собеседника
    \item Не следует делать замечания в ходе беседы
    \item В ходе переговоров возможны уточнения
    \item Избегайте перефразирования
    \item Не допускайте влияния своего эмоционального состояния на ход переговоров
    \item Необходимо правильно выбирать момент подведения промежуточных итогов
    \item Иногда переговоры ведутся нечестно. Так делать не следует
\end{enumerate}

\subsection{Деловое совещание}

Деловое совещание - общепринятая форма делового общения, когда обсуждаются вопросы, проблемы, которые требуют коллективного решения. В этом участвует коммуникативный лидер. На совещании рассматриваются только темы, которые не удаётся решить отдельным специалистам в рабочее время.

Обязательна повестка дня - письменный документ с темой совещания, перечнем обсуждаемых вопросов, время начала и окончания совещания, место проведения и должности докладчиков (обычно 6-7 человек), время рассмотрения каждого вопроса.

Лучше, чтобы совещание вёл не руководитель, а специалист, наиболее компетентный в проблемной области.

Рассаживать участников следует так, чтобы они видели глаза, лицо, мимику, жесты друг друга.

\subsection{Конференция}

Делится на дилерские, маркетинговые, отраслевые и т.д.

Следует разработать деловую креативную концепцию, арендовать помещение и оборудование, создать индивидуальный стиль, \missed{}, обеспечить проживание и питание участников.

Дилерское мероприятие - выездные конференции, выставки, съезды, \missed{}\dots. Готовятся долго.

Тренинги - краткосрочные мероприятия обучающей направленности.

Выставки - с целью демонстрации продукта, имиджа компании, поддержки её репутации. Требуется подходящее помещение. После устраивают фуршеты, шоу, \dots. Могут быть постоянными или разовыми. Могут быть и для нескольких компаний.

\subsection{Презентация}

Презентация - для нахождения партнёров и клиентов, продвижения продукции. Требуется помещение с оборудованием.

\subsection{Деловые приёмы}

Деловые приёмы - светские мероприятия для налаживания контактов, связей с другими компаниями, решения определённых вопросов.

Бизнес-форумы. Масштабные мероприятия. Должна быть площадка, реклама, организован трансфер, проживание гостей.

\section{Деловые отношения в работе. Лекция от 27.04.2023 г.}

Деловое общение в значительной степени зависит от личных отношений участников коммуникаций, а также от морально-этических норм в коллективе.

Коллектив - группа людей, объединённых общими целями и мотивами \missed{}. Всё чаще называют бизнес-команда - ограниченное количество психологически совместимых профессионалов, способных \textit{решать} задачи и добиваться высоких результатов в \textit{профессиональной деятельности}.

Важную роль в карьере менеджера играют краткосрочные перспективы и амбициозность.

Руководству компании следует проводить регулярные аудиторские проверки, тем самым повышая для нечестных менеджеров риск быть разоблачённым.

\missed{} климат формируется на основе этических норм и принципов.

Также в компании есть понятие рабочая группа - объединение из нескольких человек, которые взаимодействуют друг с другом для эффективного выполнения задачи.

\subsection{Специфика деловых отношений}

Работает по следующему принципу:
\begin{itemize}
    \item Чёткое представление поставленной задачи
    \item Осознание своей роли в решении проблемы
    \item Объединение усилий каждого для достижения результата
    \item Поиск решений
    \item Подготовка мероприятий, решающих проблему
\end{itemize}

В любой рабочей группе есть психология.

Основа успеха любого коллектива - сотрудничество и взаимопомощь.
Взаимодействия в группе зависят от психологического климата, способов общения, общего мнения и настроения.

Психологический климат - настроение коллектива, моральная и психологическая атмосфера, влияющая на взаимоотношения её участников. Здоровый климат повышает производительность работы, а неблагоприятный на 20\% снижает и вызывает нарушения безопасности.

Типы взаимоотношений:
\begin{itemize}
    \item Приказание (Руководитель непрофессионален и не способен к ответственности)
    \item Внушение (Подчинённый несамостоятелен, но готов взять на себя ответственность)
    \item Участие (Подчинённый способен к самостоятельному выполнению задания и от руководителя требуется определённая поддержка и совместное принятие решений)
    \item Передача полномочий (У руководителя наименьшее участие, подсинённый достиг высокого уровня профессионализма)
\end{itemize}

Типы темперамента (холерик, меланхолик, сангвиник, флегматик) следует учитывать.

\subsection{Лидерство}

В любом коллективе есть лидер.

Лидерство - способность оказывать влияние в группе для достижения цели. Лидер берёт ответственность за выполнение работ.

\begin{itemize}
    \item Оценка лидера членами группы часто не совпадает
    \item Неточные цели приводят к конфликтам
\end{itemize}

Компоненты лидерства:
\begin{itemize}
    \item Деловое лидерство - способность руководить, сплачивать коллектив, решать определённые задачи.
    \item Эмоциональное лидерство - вызывает у людей уверенность.
    \item Информационное лидерство - человек-эрудит, может объяснить и помочь.
\end{itemize}

Руководители совмещают 6 основных ролей:
\begin{enumerate}
    \item Хозяин
    \item Предприниматель
    \item Дипломат
    \item Менеджер
    \item Профессионал
    \item Член команды
\end{enumerate}

\missed{} и направлении их усилий для выполнения организационных задач.

Моральный климат в коллективе в значительной мере формируется в деловом общении с подчинёнными. Наблюдая за руководителем, они выясняют, какие действия обязательны, а какие нежелательны в коллективе.
\missed{} большинство людей чувствуют себя нравственно незащищёнными.

Руководитель, который ставит определённые цели всегда будет стремиться сделать сплочённый коллектив. Человек почувствует себя психологически комфортно, когда пройдёт \textit{интеграцию} в коллективе.

Руководитель может \textit{ругать} подчинённых.

Не следует критиковать подчинённого при других подчинённых. Также нельзя давать советы подчинённым в личных делах.

Руководитель должен оказывать поддержку коллективу, доверять сотрудникам и защищать их.

Руководитель ответственен за перспективы своей организации.

Подчинённый может спорить с руководителем, но в рамках этики делового общения.

2 вида лидера: формальный и неформальный.

\end{document}